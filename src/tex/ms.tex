% Define document class
\documentclass[modern]{aastex631}
\usepackage{showyourwork}

\newcommand{\sbu}{Department of Physics and Astronomy, Stony Brook University, Stony Brook NY 11794, USA}
\newcommand{\cca}{Center for Computational Astrophysics, Flatiron Institute, New York NY 10010, USA}

\newcommand{\dd}{\mathrm{d}}
\newcommand{\order}[1]{\mathcal{O}\left( #1 \right)}

\newcommand{\bNS}{b_\mathrm{NS}}
\newcommand{\phiNS}{\phi_\mathrm{NS}}
\newcommand{\rNS}{r_\mathrm{NS}}

\newcommand{\rSchw}{r_\mathrm{Schw}}

% Begin!
\begin{document}

% Title
\title{Lensing in the Schwarzschild Spacetime Applicable to Neutron Stars}

% Author list
\author[0000-0003-1540-8562]{Will M. Farr}
\email{wfarr@flatironinstitute.org}
\affiliation{\sbu}
\affiliation{\cca}

% Abstract with filler text
\begin{abstract}
    I describe the effects of gravitational lensing on the image of the surface
    of a spherical object whose surface is well outside the Schwarszchild
    horizon.
\end{abstract}

% Main body with filler text
\section{Introduction}
\label{sec:intro}

\section{Geodesics}
\label{sec:geodesics}

The Schwarzchild line element in standard coordinates and natural units is 
\begin{equation}
\dd s^2 = g_{\mu \nu} \dd x^\mu \dd x^\nu = - \left( 1 - \frac{2}{r} \right) \dd t^2 + \frac{1}{1 - \frac{2}{r}} \dd r^2 + r^2 \left( \dd \theta^2 + \sin^2 \theta \dd \phi^2\right);
\end{equation}
$t$ is the clock time of an observer at $r = \infty$, and $4\pi r^2$ is the area
of the constant-$r$, constant-$t$ hypersurfaces.  The Schwarzschild radius (the
coordinate of the event horizon) is $\rSchw = 2$ in these units.

The metric has two killing vectors, $\partial_t$ and $\partial_\phi$, since it
is independent of the corresponding coordinates.  Write
\begin{equation}
\partial_t = \xi^\mu \partial_\mu, \qquad \partial_\phi = \zeta^\mu \partial_\mu;
\end{equation}
then the only non-zero component of $\xi^\mu$ is $\xi^t = 1$, and the only
non-zero component of $\zeta^\mu$ is $\zeta^\phi = 1$.  If $p^\mu$ is the
momentum associated with a geodesic,
\begin{equation}
p^\mu = \frac{\dd \chi^\mu(\lambda)}{\dd \lambda},
\end{equation}
with $\lambda$ either proper time (for time-like geodesics) or an affine
parameter (null geodesics), and $\chi^\mu(\lambda)$ the coordinates of the
points on the geodesic, then N\"{o}ther's Theorem gives 
\begin{eqnarray}
\label{conserved}
g_{\mu \nu} \xi^\mu p^\nu = - \left( 1 - \frac{2}{\chi^r} \right) p^t & = & \textnormal{const} \equiv - e,\\
g_{\mu \nu} \zeta^\mu p^\nu = (\chi^r)^2 \sin^2 (\chi^\theta) p^\phi & = & \textnormal{const} \equiv l
\end{eqnarray}
along any geodesic.  ($e$ and $l$ are the energy and angular momentum---per unit
mass, if appropriate---of the particle traveling the geodesic measured by an
observer at infinity.)

For a null geodesic, we also have the normalization condition 
\begin{equation}
\label{normalization}
g_{\mu \nu} p^\mu p^\nu = 0.
\end{equation}
Exploiting the rotational symmetry of the $\theta$-$\phi$ subspace, we can,
without loss of generality, consider only geodesics which have $\chi^\theta
(\lambda) = \pi /2$.  In this case, Equation \ref{normalization} becomes
\begin{equation}
-\frac{e^2}{1 - \frac{2}{\chi^r}} + \frac{l^2}{(\chi^r)^2} + \frac{1}{1-\frac{2}{\chi^r}} \left( \frac{\dd \chi^r}{\dd \lambda} \right)^2 = 0.
\end{equation}
Rescaling the affine parameter, we obtain
\begin{equation}
\label{radial-path}
\frac{\dd \chi^r(s)}{\dd s} = \pm \sqrt{1 - \frac{b^2\left(1 - \frac{2}{\chi^r(s)} \right)}{(\chi^r(s))^2}},
\end{equation}
with $b \equiv l/e$ called the ``impact parameter''\footnote{$b$ is so named
because it gives the distance at infinity between the geodesic and the nearest
parallel line running radially from the center of the star.} of the trajectory.
This equation and Equation \ref{conserved} completely determine a trajectory
given $l$ and $e$.  

\section{Intersection with the Surface}
\label{sec:surface}

For a given $b$, Equation \ref{radial-path} implies a point of closest approach
to the mass at 
\begin{equation}
\frac{\dd \chi^r(s_0)}{\dd s} = 0 = 1 - \frac{b^2 \left( 1 - \frac{2}{\chi^r\left(s_0\right)} \right)}{(\chi^r(s_0))^2}.
\end{equation}
This gives a cubic equation for $\chi^r(s_0)\equiv r_0$, whose solution is not
terribly illuminating.  

It is of interest to determine the impact parameter, $\bNS$, of the ``grazing
ray'' whose point of closest approach, $r_0$, is equal to the radius of the
neutron star, $\rNS$.  Noting $\rNS \gg 2$ for physical neutron stars, we have 
\begin{equation}
     \bNS^2 = \frac{\rNS^2}{1 - \frac{2}{\rNS}} = \rNS^2 \left( 1 + \frac{2}{\rNS} + \frac{4}{\rNS^2} + \ldots \right),
\end{equation}
or 
\begin{equation}
    \bNS = \rNS \left( 1 + \frac{1}{\rNS} + \frac{3}{2 \rNS^2} + \ldots \right).
\end{equation}
That the impact parameter of the grazing ray is \emph{larger} than the neutron
star radius is a result of gravitational lensing.  It is interesting that the
leading order addition to the impact parameter is one half the Schwarzschild
radius.

We wish to determine the longitude of the intersection of a ray with impact
parameter $b < \bNS$ and the surface; this is $\chi^\phi$, the $\phi$ coordinate
of the null geodesic, when the ray intersects the surface.  To do this, it will
be more convenient to parameterize the trajectory by $\chi^r$. Using the
trajectory derived in the last section (and remembering the scaling on the
affine parameter $s$), we have 
\begin{equation}
\frac{\dd \chi^\phi}{\dd \chi^r} = \frac{\dd \chi^\phi/\dd s}{\dd \chi^r/\dd s} = \frac{-b}{(\chi^r)^2 \sqrt{1 - \frac{b^2 \left(1 - \frac{2}{\chi^r} \right)}{(\chi^r)^2}}}.
\end{equation}
Thus, the longitude at ``impact'' is 
\begin{equation}
\chi^\phi\left( \rNS \right) = \int_{\rNS}^{\infty} \dd u \, \frac{b}{u^2 \sqrt{1 - \frac{b^2\left(1 - \frac{2}{u} \right)}{u^2 }}}.
\end{equation}
(The impact parameter of the grazing ray is $\bNS = \rNS\left( 1 + 1/\rNS +
\ldots \right)$, so the argument of the square root approaches zero for the
grazing ray at impact, but otherwise is guaranteed to be positive.)

It will be convenient to transform variables from $u$ to $w = u/\rNS$, whence 
\begin{equation}
    \chi^\phi\left( \rNS \right) = \int_{1}^{\infty} \dd w \, \frac{b}{\rNS w^2 \sqrt{1 - \frac{b^2\left(1 - \frac{2}{\rNS w} \right)}{\rNS^2 w^2}}}.
\end{equation}
It is also convenient to make the substitution 
\begin{equation}
    b = x \bNS = x \rNS \left( 1 + \frac{1}{\rNS} + \frac{3}{2 \rNS^2} + \ldots \right),
\end{equation}
whence $-1 < x < 1$ for rays that impact the neutron star surface. This produces 
\begin{equation}
    \chi^\phi\left( \rNS \right) = \int_1^\infty \dd w \, \frac{x \left( 1 + \frac{1}{\rNS} + \ldots \right)}{w^2 \sqrt{1 - \frac{x^2 \left( 1 + \frac{2}{\rNS} + \ldots \right)\left( 1 - \frac{2}{\rNS w} \right)}{w^2}}}.
\end{equation}

Expanding to leading order in $1/\rNS$, we obtain 
\begin{equation}
    \chi^\phi\left( \rNS \right) = \arcsin x + \frac{2 \left( 1 - \sqrt{1-x^2} \right)}{x \rNS} + \ldots.
\end{equation}
We see that the grazing ray ($x = 1$) impacts the stellar surface at longitude 
\begin{equation}
    \phiNS = \frac{\pi}{2} + \frac{2}{\rNS} + \ldots = \frac{\pi}{2} + \frac{\rSchw}{\rNS} + \ldots
\end{equation}

\bibliography{bib}

\end{document}
